\documentclass[conference]{IEEEtran}

\usepackage{cite}
\usepackage{amsfonts,amsmath,amssymb,amsthm,booktabs,color,enumitem,graphicx}


% correct bad hyphenation here
\hyphenation{op-tical net-works semi-conduc-tor}


\begin{document}

\title{Deployment pipeline}

\author{\IEEEauthorblockN{Olli Rissanen}
\IEEEauthorblockA{Department of Computer Science\\
University of Helsinki\\
Helsinki, Finland\\
Email: olli.rissanen@helsinki.fi}}

% make the title area
\maketitle


\begin{abstract}
Currently more and more software companies are moving to lean practices, which often include shorter delivery cycles and thus shorter feedback loops. However, to achieve continuous customer feedback and to eliminate work that doesn't generate value, even shorter cycles are required. In continuous deployment the software functionality is deployed continuously at customer environment. This process includes both automated builds and automated testing, but also automated deployment. This adds more elements to the development pipeline, which often in a lean team consists of a version control system and a continuous integration server. Automating the whole process minimizes the time required for implementing new features in software, and allows for faster customer feedback. However, adopting continuous deployment doesn't necessarily mean that more value is created for the customer. 

According to multiple authors,  
\end{abstract}
% IEEEtran.cls defaults to using nonbold math in the Abstract.
% This preserves the distinction between vectors and scalars. However,
% if the conference you are submitting to favors bold math in the abstract,
% then you can use LaTeX's standard command \boldmath at the very start
% of the abstract to achieve this. Many IEEE journals/conferences frown on
% math in the abstract anyway.

% no keywords




% For peer review papers, you can put extra information on the cover
% page as needed:
% \ifCLASSOPTIONpeerreview
% \begin{center} \bfseries EDICS Category: 3-BBND \end{center}
% \fi
%
% For peerreview papers, this IEEEtran command inserts a page break and


% creates the second title. It will be ignored for other modes.
\IEEEpeerreviewmaketitle

%Perusteet

%Kiinnostava
%Innostava
%Tehtävissä

%Määrittele
% Tutkimusongelma, siis mitä haluat ymmärtää tai selvittää
% Tavoitteet
% Menetelmät
% Rajaus

%Mahdollisia toteutustapoja
% Vertaile kahta tai useampaa lähestymistapaa
% Etsi empiiristä todistusaineistoa jonkin väitteen puolesta tai sitä vastaan
% Kuvaa tapausyrityksen tilanne ja vertaile kirjallisuuteen
%

%Haasteet

%In fact, one sign of a good application architecture is that it allows the application
%to be run without much trouble on a development machine.

\section{Introduction} %why is this problem interesting?
Continuous deployment is an extension to continuous integration, where the software functionality is deployed frequently at customer environment. While continuous integration defines a process where the work is automatically built, tested and frequently integrated to mainline \cite{fowler2006continuous}, often multiple times a day, continuous deployment adds automated acceptance testing and deployment. The purpose of continuous deployment is that as the deployment process is completely automated, it reduces human error, documents required for the build and increases confidence that the build works \cite{cdbook}. %Explain pipeline here

An important part of continuous deployment is the deployment pipeline. A deployment pipeline can be loosely defined as a consecutively executed set of validations that a software has to pass such before it can be released. The deployment pipeline is therefore an automated implementation of the application's build, deploy, test and release process \cite{cdbook}. Common components of the deployment pipeline are a version control system and an automated test suite.

In an agile process software release is done in periodic intervals \cite{cockburn2002agile}. Compared to waterfall model it introduces multiple releases throughout the development. Continuous deployment, on the other hand, attemps to keep the software ready for release at all times during development process \cite{cdbook}. Instead of stopping the development process and creating a build as in an agile process, the software is continuously deployed to customer environment. This doesn't mean that the development cycles in continuous deployment are shorter, but that the development is done in a way that makes the software always ready for release.

It should also be made clear that continuous delivery differs from continous deployment. Refer to Fig. \ref{fig1} for a visual representation of differences in continuous integration, delivery and deployment. Both include automated deployment to a staging environment. Continuous deployment includes deployment to a production environment, while in continuous delivery the deployment to a production environment is done manually. The purpose of continuous delivery is to prove that every build is proven deployable \cite{cdbook}. While it necessarily doesn't mean that teams release often, keeping the software in a state where a release can be made instantly is often seen beneficial.

Adopting a continuous deployment process doesn't necessarily mean that more value is delivered to a customer. Avinash Kaushik states in his Experimentation and Testing primer \cite{kaushik} that "80\% of the time you/we are wrong about what a customer wants". Mike Moran also found similar results in his book Do It Wrong Quickly, stating that Netflix considers 90\% of what they try to be wrong. A way to tackle this issue is to adopt a process of continuous experimentation, where the entire R&D system responds and acts based on instant customer feedback, and where actual deployment of software functionality is seen as a way of experimenting and testing what the customer needs \cite{olsson2012climbing}.
 
Continuous deployment attempts to deliver an idea to users as fast as possible. Continuous experimentation instead attempts to validate that it is, in fact, a good idea. In continuous experimentation the organisation runs controlled experiments to guide the R&D process. %Mitä continous experimentation tekee?

Continuous experimentation requires operational changes to the deployment pipeline. 

In this paper we're investigating

This paper is organized as follows.

\begin{figure}[!t]
	\centering
	\includegraphics[width=3.5in]{developmentprocess.jpg}
	\caption{Architecture of the development process \cite{cdbook}.}
	\label{fig1}
\end{figure}


\section{Methods} %how did I find the papers? %TODO: UPDATE THIS


%Kerro tarkemmin kuin vaan mitä hakuja tehtiin
Searches were performed using the keywords shown in Table I. The searches were performed during February and March 2014 using IEEE Xplore (http://ieeexplore.ieee.org/‎) and Google Scholar (http://scholar.google.com/) search engines. Final papers were downloaded from the publishers web sites, if available. Plenty of research concerning software release management, release planning and iterative release planning were found, but the focus on architectural qualities appeared sparsely.  

\begin{center}
\begin{table}
    \caption{Keywords used for searching research materials.}
    \begin{tabular}{ | p{2cm} | p{2cm} | p{3.5cm} |}
    \hline
    Search string & Search engine & Article \\ \hline
    software release management & Google Scholar & Software Release Management \\ \hline
	release planning architecture & IEEE Xplore & Importance of Software Architecture during Release Planning \\ \hline
	deployment production line & IEEE Xplore & The deployment production line \\ \hline
	iterative release planning architecture & IEEE Xplore & Analysis and Management of Architectural Dependencies in Iterative Release Planning \\ \hline
    \end{tabular}
    \end{table}
\end{center}

As searches regarding architecture in continuous delivery and deployment returned no results, a decision was made to instead focus on research related to architectural features and issues in releases, and then attempting to apply those findings to continuous deployment. Articles regarding release management, release planning, iterative release planning and deployment pipeline were chosen for this purpose.

\section{Results} %Pure analysis on the papers

Lean startup \cite{ries2011lean}

Stairway to heaven Olsson \cite{olsson2012climbing}

Building products as innovative \cite{bosch2012building}

\subsection{Pipeline}

"A deployment pipeline is an automated implementation of your application's build, deploy, test and release process" \cite{cdbook}

When deployments aren’t fully automated, errors will occur every time they are performed \cite{cdbook}.

\subsection{Experimentation}
-Experimentation in practice is often limited to optimizing narrow aspecs of the front-end through A/B testing, and is not systematically applied \cite{bosch2012building}.

Seven pitfalls \cite{crook2009seven}


%Split this to themes instead of papers

\section{Discussion} %Own speculation

What kind of elements need to be added to the deployment pipeline based on these papers?

\section{Future research} %Self-explanatory
\section{Conclusion} %Short wrap-up. Contains the key points


\bibliography{IEEEabrv,references}{}
\bibliographystyle{IEEEtran}
% trigger a \newpage just before the given reference
% number - used to balance the columns on the last page
% adjust value as needed - may need to be readjusted if
% the document is modified later
%\IEEEtriggeratref{8}
% The "triggered" command can be changed if desired:
%\IEEEtriggercmd{\enlargethispage{-5in}}

% references section

% can use a bibliography generated by BibTeX as a .bbl file
% BibTeX documentation can be easily obtained at:
% http://www.ctan.org/tex-archive/biblio/bibtex/contrib/doc/
% The IEEEtran BibTeX style support page is at:
% http://www.michaelshell.org/tex/ieeetran/bibtex/

% argument is your BibTeX string definitions and bibliography database(s)
%\bibliography{IEEEabrv,../bib/paper}
%
% <OR> manually copy in the resultant .bbl file
% set second argument of \begin to the number of references
% (used to reserve space for the reference number labels box)
%\begin{thebibliography}{1}

%\bibitem{IEEEhowto:kopka}
%H.~Kopka and P.~W. Daly, \emph{A Guide to \LaTeX}, 3rd~ed.\hskip 1em plus
%  0.5em minus 0.4em\relax Harlow, England: Addison-Wesley, 1999.

%\end{thebibliography}




% that's all folks
\end{document}


