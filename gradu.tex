% --- Template for thesis / report with tktltiki2 class ---
% 
% last updated 2013/02/15 for tkltiki2 v1.02

\documentclass[english]{tktltiki2}

% tktltiki2 automatically loads babel, so you can simply
% give the language parameter (e.g. finnish, swedish, english, british) as
% a parameter for the class: \documentclass[finnish]{tktltiki2}.
% The information on title and abstract is generated automatically depending on
% the language, see below if you need to change any of these manually.
% 
% Class options:
% - grading                 -- Print labels for grading information on the front page.
% - disablelastpagecounter  -- Disables the automatic generation of page number information
%                              in the abstract. See also \numberofpagesinformation{} command below.
%
% The class also respects the following options of article class:
%   10pt, 11pt, 12pt, final, draft, oneside, twoside,
%   openright, openany, onecolumn, twocolumn, leqno, fleqn
%
% The default font size is 11pt. The paper size used is A4, other sizes are not supported.
%
% rubber: module pdftex

% --- General packages ---

\usepackage[utf8]{inputenc}
\usepackage[T1]{fontenc}
\usepackage{lmodern}
\usepackage{microtype}
\usepackage{amsfonts,amsmath,amssymb,amsthm,booktabs,color,enumitem,graphicx}
\usepackage[pdftex,hidelinks]{hyperref}

% Automatically set the PDF metadata fields
\makeatletter
\AtBeginDocument{\hypersetup{pdftitle = {\@title}, pdfauthor = {\@author}}}
\makeatother

% --- Language-related settings ---
%
% these should be modified according to your language

% babelbib for non-english bibliography using bibtex
\usepackage[fixlanguage]{babelbib}

% add bibliography to the table of contents
\usepackage[nottoc]{tocbibind}

% --- Theorem environment definitions ---

\newtheorem{thm}{Theorem}
\newtheorem{lem}[thm]{Lemma}
\newtheorem{cor}[thm]{Corollary}

\theoremstyle{definition}
\newtheorem{definition}[thm]{Definition}

\theoremstyle{remark}
\newtheorem*{remark}{Remark}


% --- tktltiki2 options ---
%
% The following commands define the information used to generate title and
% abstract pages. The following entries should be always specified:

\title{Extending the “Development Pipeline” Towards Continuous Deployment and Continuous Experimentation: A Case Study in the B2B Domain}
\author{Olli Rissanen}
\date{\today}
\level{Master's thesis}
\abstract{Currently more and more software companies are moving to lean practices, which often include shorter delivery cycles and thus shorter feedback loops. However, to achieve continuous customer feedback and to eliminate work that doesn't generate value, even shorter cycles are required. In continuous deployment the software functionality is deployed continuously at customer environment. This process includes both automated builds and automated testing, but also automated deployment. Automating the whole process minimizes the time required for implementing new features in software, and allows for faster customer feedback. However, adopting continuous deployment doesn't necessarily mean that more value is created for the customer. While continuous deployment attempts to deliver an idea to users as fast as possible, continuous experimentation instead attempts to validate that it is, in fact, a good idea. In a state of continuous experimentation, the entire R\&D process is guided by controlled experiments and feedback. In it's core continuous experimentation consists of a design-execute-analyse loop, where hypotheses are selected based on business goals and strategies, experiments are executed with partial implementations and data collection tools and finally the results are analyzed to validate the hypothesis. In this paper we're ..}

% The following can be used to specify keywords and classification of the paper:

\keywords{Continuous delivery, Continuous experimentation, Development pipeline}

% classification according to ACM Computing Classification System (http://www.acm.org/about/class/)
% This is probably mostly relevant for computer scientists
% uncomment the following; contents of \classification will be printed under the abstract with a title
% "ACM Computing Classification System (CCS):"
% \classification{}

% If the automatic page number counting is not working as desired in your case,
% uncomment the following to manually set the number of pages displayed in the abstract page:
%
% \numberofpagesinformation{16 pages + 10 appendix pages}
%
% If you are not a computer scientist, you will want to uncomment the following by hand and specify
% your department, faculty and subject by hand:
%
\faculty{Faculty of Science}
\department{Department of Computer Science}
\subject{Computer Science}
%
% If you are not from the University of Helsinki, then you will most likely want to set these also:
%
\university{University of Helsinki}
% \universitylong{HELSINGIN YLIOPISTO --- HELSINGFORS UNIVERSITET --- UNIVERSITY OF HELSINKI} % displayed on the top of the abstract page
\city{Helsinki}
%


\begin{document}

% --- Front matter ---

\frontmatter      % roman page numbering for front matter

\maketitle        % title page
\makeabstract     % abstract page

\tableofcontents  % table of contents

% --- Main matter ---

\newpage

It’s hard to argue that Tiger Woods is pretty darn good at what he does. But even he is not perfect. Imagine if
he were allowed to hit four balls each time and then choose the shot that worked the best. Scary good.
-- Michael Egan, Sr. Director, Content Solutions, Yahoo (Egan, 2007)

\mainmatter       % clear page, start arabic page numbering



%case study
% Relate the theory to a practical situation; for example, apply the ideas 
%and knowledge discussed in the coursework to the practical situation 
%at hand in the case study. 
 
% Identify the problems 
% Select the major problems in the case 
% Suggest solutions to these major problems 
% Recommend the best solution to be implemented 
% Detail how this solution should be implemented

\section{Introduction}
-GOAL OF THE THESIS
-motivation
-research question
-approach

>analyze state of the practice
>specify problem and goals 
>analyze state of the art
>state hypotheses
>derive solution idea

\section{Related work}
-continuous delivery
-continuous experimentation
-state of the art practices
    -short summary

\subsection{Continuous delivery}
Continuous deployment is an extension to continuous integration, where the software functionality is deployed frequently at customer environment. While continuous integration defines a process where the work is automatically built, tested and frequently integrated to mainline \cite{fowler2006continuous}, often multiple times a day, continuous deployment adds automated acceptance testing and deployment. The purpose of continuous deployment is that as the deployment process is completely automated, it reduces human error, documents required for the build and increases confidence that the build works \cite{cdbook}. %Explain pipeline here

An important part of continuous deployment is the deployment pipeline, which is an automated implementation of an application's build, deploy, test and release process \cite{cdbook}. A deployment pipeline can be loosely defined as a consecutively executed set of validations that a software has to pass such before it can be released. Common components of the deployment pipeline are a version control system and an automated test suite.

In an agile process software release is done in periodic intervals \cite{cockburn2002agile}. Compared to waterfall model it introduces multiple releases throughout the development. Continuous deployment, on the other hand, attemps to keep the software ready for release at all times during development process \cite{cdbook}. Instead of stopping the development process and creating a build as in an agile process, the software is continuously deployed to customer environment. This doesn't mean that the development cycles in continuous deployment are shorter, but that the development is done in a way that makes the software always ready for release.

It should also be made clear that continuous delivery differs from continous deployment. Refer to Fig. \ref{fig1} for a visual representation of differences in continuous integration, delivery and deployment. Both include automated deployment to a staging environment. Continuous deployment includes deployment to a production environment, while in continuous delivery the deployment to a production environment is done manually. The purpose of continuous delivery is to prove that every build is proven deployable \cite{cdbook}. While it necessarily doesn't mean that teams release often, keeping the software in a state where a release can be made instantly is often seen beneficial.

\subsection{Experimentation}
An experiment is essentially a procedure to confirm the validity of a hypothesis. In software engineering context, experiments attempt to answer questions such as which features are necessary for a product to succeed, what should be done next and which customer opinions should be listened to. According to Jan Bosch, "The faster the organization learns about the customer and the real world operation of the system, the more value it will provide" \cite{bosch2012building}. Most organizations have many ideas, but the return-on-investment for many may be unclear and the evaluation itself may be expensive \cite{kohavi2007practical}. I

In Lean startup methodology \cite{ries2011lean} experiments consist of Build-Measure-Learn cycles, and are tightly connected to visions and the business strategy. The purpose of a Build-Measure-Learn cycle is to turn ideas into products, measure how customers respond to the product and then to either pivot or persevere the chosen strategy. The cycle starts with forming a hypothesis and building a minimum viable product (MVP) with tools for data collection. Once the MVP has been created, the data is analyzed and measured in order to validate the hypothesis. To persevere with a chosen strategy means that the experiment proved the hypothesis correct, and the full product or feature can is implemented. However, if the experiment proved the hypothesis wrong, the strategy is changed based on the implications of a false hypothesis.

Jan Bosch has widely studied continuous experimentation, or innovation experiment systems, as a basis for development. The primary issue he found is that "experimentation in online software is often limited to optimizing narrow aspects of the front-end of the website through A/B testing and inconnected, software-intensive systems experimentation, if applied at all, is ad-hoc and not systematically applied" \cite{bosch2012building}. The author realized that for different development stages, different techniques to implement experiments and collect customer feedback exist. Bosch also introduces a case study in which a company, Intuit, adopted continuous experimentation and has increased both the performance of the product and customer satisfaction.

Fig. \ref{fig4} introduces different stages and scopes for experimentation. For each stage and scope combination, an example technique to collect product performance data is shown. As startups often start new products and older companies instead develop new features, experiments must be applied in the correct context. Bosch states that for a new product deployment, putting a minimal viable product as rapidly as possible in the hands of customers is essential \cite{bosch2012building}. After the customers can use the product, it is often not yet monetizable but is still of value to the customer. Finally, the product is commercially deployed and collecting feedback is required to direct R\&D investments to most valuable features.
\begin{figure}[H]
	\centering
	\includegraphics[width=3.5in]{bosch.jpg}
	\caption{Scopes for experimentation\cite{bosch2012building}.}
	\label{fig4}
\end{figure}

\subsection{Continuous experimentation}
Continuous deployment attempts to deliver an idea to users as fast as possible. Continuous experimentation instead attempts to validate that it is, in fact, a good idea. In continuous experimentation the organisation runs controlled experiments to guide the R\&D process. The development cycle in continuous experimentation resembles the build-measure-learn cycle of lean startup \cite{ries2011lean}. The process in continuos experimentation is to first form a hypothesis based on a business goals and customer "pains" \cite{bosch2012building}. After the hypothesis has been formed, quantitative metrics to measure the hypothesis must be decided. After this a minimum viable product can be developed and deployed, while collecting the required data. Finally, the data is analyzed to attempt to validate the hypothesis.

As the experiments are run in a regular fashion, integrating experiments to the deployment pipeline should be considered. This requires changing the development process in such fashion that functionality is developed based on some actual data. The components required to support continuous experimentation include tools to assign users to treatment and control groups, tools for data logging and storing, and analytics tool for conducting statistical analyses.

\subsection{State of the art}

\section{State of the practice}	
%an overview of the entire process, then refining the parts that are relevant for the problems 
Steeri is a mid-sized company of 80 employees, focusing in managing and improving customer data usage. This includes CRM systems, business intelligence solutions, customer dialog and data integration. Steeri has created Customer data management (CDM) and Customer dialog (Dialog) products, which are developed by two different teams. 

These two teams develop in an agile manner, but the continuous integration process \cite{fowler2006continuous} and short feedback cycles aren't achieved yet. 

The development isn't at all distributed, and no external party affects the development process. 

\subsection{iSteer Contact backlog tool}

\subsubsection{Overview}

\subsubsection{Customer backlog}
Work items are first added to customer backlogs in iSteer Contact. In this phase they might be just initial skeletons and do not need to contain all needed information. The work items are added by product or business owners or project managers. At this phase the created backlog items are not yet visible in team backlogs in iSteer Contact. The idea is to list them in the customer backlog as early as possible and start refining the requirements collaboratively both offline and online with the help of tools such as Chatter. Chatter can be used to discuss a single story or the complete backlog.

When the backlog item is ready for the development team to start working on it, it should contain at least the following information:
A descriptive name
Specifications that explain both the technical side and the business side
Agilefant link four hour reporting

The backlog item is added to team backlog by selecting the “show in team backlog” -option.

\subsubsection{Team backlog}
Team backlog is a view to all stories assigned to one selected team. It is a tool to collect and organize stories from all customer/project backlogs without cloning details into multiple places. It’s just a view of all “in team backlog” stories that are not yet completed and are assigned to the selected team.

Completed stories related to one team can be found from the report (link in team detail page)
Team backlog items (stories) are prioritized and estimated once a week (Tuesday) and new stories will be moved to Trello when accepted by the development team.

Stories can be moved to Trello by project managers or product owners but they must be checked by team leader (Juha) before moving to 

\subsection{Using Trello}

\subsubsection{Overview}
Trello is a project management application that uses Kanban to control the production chain from development to release. Kanban attempts to limit the work currently in progress by establishing an upper limit of tasks in the backlog, thus avoiding overloading of the team. Trello consists of multiple boards, each representing a project or a development team. A board consists of a list of columns, and each column consists of cards. Columns each contain a list of tasks, and cards progress from one column to the next when each task has been completed. A card is essentially a task, which is added by the Backlog Owner, and can be checked out by a developer. In Steeri, the columns used are Sprint Backlog, In Progress, Review, Ready, Verified and Done. 

\subsubsection{Sprint Backlog}
The Backlog Owner (Juha) moves the cards that have the highest priority into the sprint backlog. The sprint backlog should always contain enough cards so that whenever a developer completes a task something new is available in the backlog.

\subsubsection{In progress}
The actual development work is done in this stage. The actor here is the developer. The checklist to move a card into review stage is:
·       All tasks are complete
·       Changes are pushed to a feature-specific branch
·       Unit tests are written and pass in the CI server
·       Feature is well-written and does not need refactoring
·       Pull request is created and a link is added to the comment field
·       Feature has been documented as needed
If the feature needs refactoring a task list must be created and the card moved back to In Progress column.

\subsubsection{Review}
Here other developers review the new code and deploy it to a development environment. Checklist for moving the card into ready stage is:
·       Pull request is reviewed and by at least two (2) persons
·       Pull request is merged to the development branch
·       Feature is deployed to a development environment
·       Source code quality has to be good enough!
After you have reviewed the pull request leave yourself as an assignee. The second person who reviews the pull request is responsible for cleaning all the assignees and moving the feature to Ready column.
If there is a major problem in the pull request the feature should be moved back to Sprint Backlog and the yellow “Boomerang” tag added. Person who created the pull request is responsible for implementing the necessary remarks.
If only small fixes are needed, they should be implemented within the Github pull request workflow.
The second person who has reviewed and accepted the pull request is responsible for deploying the feature to a development environment.

\subsubsection{Ready}
Here the product owner verifies the new functionality in the development environment. The card can be moved to Verified if:
·       Feature has been verified by the Product Owner in the development environment
Product owner is responsible for moving the feature to the Verified column.
If the verification for the feature fails the Product Owner should move the feature back to Sprint Backlog column with the highest priority. In addition the Product Owner should add a yellow “Boomerang” label with a comment describing the results in the feature.

\subsubsection{Verified}
Here the backlog owner collects the timestamps and trello flow data. The timestamps depicts the duration it took from a card to process through the whole chain. The data is then used to analyze which columns the card spent the longest time in, and to identify the pain spots.
\subsubsection{Done}

This column simply states that the task has been completed, and should eventually be archived. There’s currently no general validation required from the customer, as the customer projects each have a different schedule and process for builds.
\subsubsection{Prioritized lists}

The backlog owner adds tasks to prioritized lists from team backlog as soon as the tasks meet the required criterias, contain the required information and are inspected by both the stakeholders and the backlog owner.

\subsection{After Trello}
Thnergefwdqa

\cite{olsson2012climbing}

-current state at steeri
    -short summary

Be sure to specify as much of the industrial context as possible. In particular,
clearly define the entities, attributes, and measures that are capturing the contextual information.

\section{Needs, problems and challenges}

\subsection{Problems} %Should address an underlying research question

TODO:
Research problem: "High-level slogan"

Research questions:

Larger context, narrow context
Subproblems
More concrete (Who is having a problem?)

-Collecting feedback in B2B domain
  In B2B domain the connection with end users might not be established properly, and the feedback can only be received from the other company's key members. However, the feedback from end users is especially important. 

  Without the data from end users, no usage data is collected all. 

  Pilot approach: Prioritize bugs based on the amount of occurrences

-Collecting feedback without a SaaS product
  In a SaaS platform it is easier to plug in measurement tools to the software, as the environment is that of the company responsible for the service. Without a SaaS environment, the data has to be collected and stored in customer environment. 

  Thus, the collection of the data takes unnecessary effort, and the whole usage data is totally missing. The product development decisions aren't therefore based on data, and are potentially less effective than decisions based on measurable, quantitative data.

-The time between the idea and implementation is too long
  Currently the ideas are not tested at all, but the effects are planned in a waterfall style and implemented thoroughly before a release. Making partial implementations could however allow the use of quantitative or qualitative data to measure whether an idea is worth implementing at all.

  Management can also better convince other internal stakeholders that a feature might be of importance

-Time dimension. If an impact is done right now, when will the effects be visible?
  Adding the correct measurement metrics to a feature also make the impact more visible.

-Conventions for production deployment. The whole pipe between the idea and implementation isn't yet completely perceived. Some parts are visible, but the entirety is still unclear.
  Forming the development pipeline to include both experiments and automated deployment makes the entirety clear from the idea to the release.

-How do we validate whether a feature is succesful or not? In B2C this is usually done by observing shifts in sales, but in our product there's no such thing.
  The measurement metric has to be chosen such that a decision can be made to judge if the feature is succesful or not.

-How to use the collected data to improve decision making?
  TODO

-Trello flow time?
  Occasional deploying takes time. Speeding it up would eventually improve flow time.

-Possible thing we can analyze is user experience, but is it enough?

\subsection{goals}
Requirements should be testable

-Continuously collect feedback from customers.
    Subgoal: Define the type of feedback to be collected
    Subgoal: Define a mechanism for feedback collection
    Subgoal: figure out how to collect feedback
    Subgoal: figure who is responsible for analyzing feedback
	Req: Feedback can be continuously collected from customers
	Req: The person responsible for collecting feedback can be identified 
-Reduce the length of the delivery cycle.
	Subgoal: Identify the components in the delivery cycle
	Subgoal: Identify what takes too long
	Req: The delivery cycle is completely perceived
	Req: The length of the delivery from idea to deployment is shortened
-Be able to measure the effectiveness of a feature.
	Subgoal: Implement measurement tools
	Subgoal: Implement data analysis tools
	Subgoal: collect data whenever a feature is deployed to measure the performance	
    Req: Data can be collected from an implemented feature 
    Req: Collected data can be analyzed to form an answer


\subsection{solutions}

\subsection{solution idea}

\subsection{Challenges respecting to continuous delivery}
-Technical implementation
-Education on the subject
-Active participation of employees

\subsection{Challenges respecting to continuous experimentation}
-Participation of every key member

\section{Research method}

Exploratory case study: Finding out what is happening, seeking new insights and generating ideas and hypotheses for new research.

Case study structure: 

1. Case study design: objectives are defined and 
the case study is planned. 
2. Preparation for data collection: procedures and 
protocols for data collection are defined. 
3. Collecting evidence: execution with data 
collection on the studied case. 
4. Analysis of collected data 
5. Reporting 

Proposal for solution


\section{Hypothesis}
\section{Methodology}
Adopt the solution
Implement & test

\section{Results}
\section{Analysis}
\section{Conclusion}


% --- References ---
%
% bibtex is used to generate the bibliography. The babplain style
% will generate numeric references (e.g. [1]) appropriate for theoretical
% computer science. If you need alphanumeric references (e.g [Tur90]), use
%
% \bibliographystyle{babalpha-lf}
%
% instead.

\bibliographystyle{babplain-lf}
\bibliography{gradu}


% --- Appendices ---

% uncomment the following

% \newpage
% \appendix
% 
% \section{Example appendix}

\end{document}
